\section{Use Cases}
\subsection{Phòng Thẩm Vấn (Interrogation Room).}

\subsubsection{Sơ đồ chức năng.}

\begin{figure}[h!]
    % Lệnh \centering dùng để căn giữa nội dung bên trong môi trường figure
    \centering
    \includegraphics[width=0.8\textwidth]{image/interrogation-room-usecase.png}
    
    \caption{Sơ đồ chức năng cho Phòng Thẩm Vấn. (Interrogation Room Usecase Diagram.)}
    
    \label{fig:interrogation-room-usecase-diagram}
\end{figure}

\subsubsection{Yêu cầu chức năng (Functional Requirements).}

\paragraph*{UCT000: Khởi tạo Giao diện Thẩm vấn (Interrogation UI Initialization)}
Use Case này mô tả quá trình thiết lập giao diện chính ngay sau khi Người chơi chọn nghi phạm.
\begin{itemize}
    \item \textbf{Actor:} Người chơi (Player).
    \item \textbf{Mô tả:} Hệ thống chuẩn bị và hiển thị giao diện thẩm vấn chính, sẵn sàng cho Người chơi tương tác.
    \item \textbf{Luồng sự kiện chính:}
    \begin{enumerate}
        \item Hệ thống nhận diện Nghi phạm được chọn.
        \item Hệ thống tải và hiển thị ba khu vực tương tác cốt lõi: Khu vực Câu hỏi Chung, Khu vực Bằng chứng, Khu vực Đối thoại (Lời khai).
        \item Hệ thống hiển thị biểu tượng truy cập nhanh của NPC Cố vấn và Nghi phạm.
        \item Hệ thống chuyển trạng thái sẵn sàng cho các hành động Thẩm vấn.
    \end{enumerate}
\end{itemize}

\paragraph*{UCA000: Thẩm Vấn Nghi Phạm (Nói chung) (General Interrogation)}
Use Case này tập trung vào các câu hỏi mở và luồng đối thoại không liên quan trực tiếp đến việc chất vấn bằng chứng.
\begin{itemize}
    \item \textbf{Actor:} Người chơi (Player), NPC Nghi phạm (Suspect).
    \item \textbf{Mục tiêu:} Sử dụng các câu hỏi chung để tìm ra mâu thuẫn hoặc thu thập thông tin mới.
    \item \textbf{Luồng sự kiện chính:}
    \begin{enumerate}
        \item Người chơi chọn một Nghi phạm (\textbf{Lựa chọn Nghi phạm}).
        \item Hệ thống thực hiện \textbf{UCT000: Khởi tạo Giao diện Thẩm vấn} (<<include>>).
        \item Người chơi \textbf{đặt các câu hỏi mở} từ Khu vực Câu hỏi Chung.
        \item Nghi phạm phản hồi (hiển thị trong Khu vực Đối thoại).
        \item Người chơi theo dõi và phát hiện mâu thuẫn (Mục tiêu).
    \end{enumerate}
\end{itemize}

\paragraph*{UCA001: Thẩm Vấn Nghi Phạm về Bằng Chứng (Evidence-Based Interrogation)}
Use Case này tập trung vào việc chất vấn Nghi phạm bằng cách sử dụng các bằng chứng đã thu thập.
\begin{itemize}
    \item \textbf{Actor:} Người chơi (Player), NPC Nghi phạm (Suspect), Bằng chứng (Evidence).
    \item \textbf{Mục tiêu:} Ép cung Nghi phạm bằng bằng chứng để làm rõ mối liên hệ của Nghi phạm với bằng chứng đó.
    \item \textbf{Luồng sự kiện chính:}
    \begin{enumerate}
        \item Người chơi \textbf{chọn một bằng chứng} từ Khu vực Bằng chứng.
        \item Hệ thống hiển thị danh sách các \textbf{câu hỏi chất vấn chuyên biệt} liên quan đến bằng chứng.
        \item Người chơi lựa chọn câu hỏi.
        \item Nghi phạm đưa ra lời khai phản hồi.
        \item Hệ thống hiển thị lời khai và các mâu thuẫn được phát hiện.
    \end{enumerate}
\end{itemize}

\paragraph*{UCI000: Tương Tác với NPC Hỗ Trợ (Mentor \& Suspect)}

Các tương tác với NPC Hỗ trợ (Mentor) và Nghi Phạm (Suspect) được chuẩn hóa thông qua giao diện trò chuyện riêng. (Nội dung giữ nguyên)

\begin{enumerate}
\item\textbf{Giao diện và Tùy chọn Tương tác}
    \begin{itemize}
        \item \textbf{Truy cập Nhanh:} Một biểu tượng hoặc nút bấm của \textbf{Người Cố Vấn (Mentor)} và \textbf{Nghi Phạm (Suspect)} sẽ luôn hiển thị trong giao diện thẩm vấn.
        \item \textbf{Tùy chọn:} Khi click vào biểu tượng của NPC, hai tùy chọn sẽ hiện lên: \textbf{"Đặt câu hỏi"} (cho cả Mentor và Suspect) và \textbf{"Xin chỉ dẫn"} (chỉ dành cho Mentor).
    \end{itemize}

\item\textbf{Chức năng "Đặt câu hỏi"}
    \begin{itemize}
        \item \textbf{Giao diện:} Quá trình trao đổi diễn ra trong một \textbf{cửa sổ trò chuyện riêng} (giống Messenger).
        \item \textbf{Danh sách Câu hỏi:} Khi người chơi chọn tùy chọn này, một danh sách các câu hỏi được phép hỏi NPC đó sẽ hiện ra.
        \item \textbf{Quy tắc Sử dụng:} Mỗi câu hỏi chỉ được sử dụng \textbf{một lần}; sau khi đặt, câu hỏi đó sẽ \textbf{biến mất} khỏi danh sách để tránh lặp lại.
    \end{itemize}
\end{enumerate}

\paragraph*{UCI001: Tương Tác Bằng Chứng với Người Cố Vấn}

Chức năng này cho phép người chơi tham vấn ý kiến chuyên môn từ Mentor về các bằng chứng thu thập được. (Nội dung giữ nguyên)

\begin{itemize}
    \item \textbf{Thao tác:}
    \begin{enumerate}
        \item Người chơi \textbf{chọn một bằng chứng} cụ thể từ khu vực Bằng chứng.
        \item Click vào biểu tượng \textbf{Mentor}.
        \item Chọn tùy chọn \textbf{"Đặt câu hỏi"} (hoặc một tùy chọn tương đương như "Tham vấn Bằng chứng").
    \end{enumerate}
    \item \textbf{Phản hồi của Mentor:} Mentor sẽ đưa ra \textbf{thông tin phân tích cơ bản}, \textbf{ý kiến chuyên môn} hoặc \textbf{bối cảnh liên quan} tới bằng chứng đó, giúp người chơi hiểu rõ hơn về giá trị pháp lý hoặc mối liên hệ của bằng chứng với vụ án.
\end{itemize}

\paragraph*{UCT001: Kết thúc Phiên Thẩm vấn (Interrogation Session Termination)}
Chức năng này mô tả quá trình Người chơi kết thúc phiên làm việc trong phòng thẩm vấn.
\begin{itemize}
    \item \textbf{Actor:} Người chơi (Player).
    \item \textbf{Mô tả:} Người chơi quyết định dừng quá trình thẩm vấn và rời khỏi giao diện.
    \item \textbf{Luồng sự kiện chính:}
    \begin{enumerate}
        \item Người chơi chọn tùy chọn "Kết thúc Phiên" hoặc "Thoát" trên giao diện.
        \item Hệ thống ghi nhận trạng thái mới nhất của lời khai, bằng chứng, và tiến trình game.
        \item Hệ thống chuyển Người chơi trở lại giao diện điều tra chung.
    \end{enumerate}
\end{itemize}
