\section{Giới thiệu vấn đề}
\subsection{Bối cảnh}
Trên thế giới, ngành Game hiện tại đang là một trong những ngành công nghiệp giải trí lớn mạnh với doanh thu hàng trăm tí USD mỗi năm. Thị trường này tràn ngập các dự án đa dạng mỗi năm, từ những game AAA (đồ họa, kinh phí khủng) của các studio lớn tới những game indie nhỏ lẻ đầy sáng tạo. Một trong những thể loại game thu hút rất nhiều sự quan tâm của người chơi là thể loại trinh thám. Với rất nhiều những tựa game đình đám từ indie đến AAA đã đạt được số doanh thu khổng lồ: dòng game Ace Attorney của Capcom bán được 13 triệu bản toàn cầu, The Room của Fireproof Games phát hành trên mobile bán được hơn 11,5 triệu bản, và Return of the Obra Dinn do một mình Lucas Pope thiết kế và hiện thực thu về tận 20 triệu USD. 

Tuy nhiên, tại Việt Nam, ngành game trinh thám vẫn còn non trẻ và gặp nhiều khó khăn:
\begin{itemize}
    \item \textbf{Thiếu vắng các sản phẩm chất lượng:} Rất ít game trinh thám Việt Nam được đầu tư bài bản về cốt truyện, gameplay và đồ họa. Các sản phẩm hiện có thường chỉ dừng lại ở mức giải đố đơn giản, thiếu chiều sâu về logic và suy luận, chưa đủ sức cạnh tranh với thị trường quốc tế.
    
    \item \textbf{Hạn chế về kinh phí và nguồn lực:} Giống như các thể loại game khác, các studio game trinh thám ở Việt Nam có kinh phí hạn chế, không đủ để đầu tư vào marketing và các dự án lớn. Điều này gây khó khăn trong việc phát triển và tiếp cận người chơi.
\end{itemize}

Thị trường Game trinh thám Việt Nam chứa đầy tiềm năng phát triển, nhưng lại đang đối mặt với nhiều rào cản để hiện thực hóa sự phát triển đó. Trong một bối cảnh mâu thuẫn như vậy, vai trò của những dự án game indie độc lập là cực kỳ quan trọng và thức thời trong việc thúc đẩy sự phát triển của thị trường này. Các dự án game indie thỏa mãn được cả hai hạn chế nêu trên của những dự án game Việt Nam hiện tại. Với kinh phí không lớn nhưng vẫn đảm bảo sự đột phá và sáng tạo, game indie là con đường khả thi để các nhà phát triển Việt tạo ra sản phẩm chất lượng, cạnh tranh được với thị trường quốc tế.

Từ thực trạng hiện tại của dòng game trinh thám Việt Nam, chúng tôi quyết định phát triển game indie The Se7enth code. Một dự án game điều tra, phá án nơi người chơi phải tương tác với các vật thể, các Non-player Character (NPC) và phải suy luận dựa trên những thông tin thu thập được để phá được các vụ án hóc búa. 

\subsection{Mục tiêu dự án}
\subsubsection{Mục đích dự án}
Xây dựng một tựa game trinh thám độc đáo và có chiều sâu, mang lại trải nghiệm điều tra và suy luận chân thực cho người chơi trên nền tảng PC. Game sẽ không chỉ đơn thuần là giải đố, mà còn là một công cụ giúp người chơi rèn luyện tư duy logic, khả năng quan sát và phân tích thông tin một cách có hệ thống.

\subsubsection{Các mục tiêu chi tiết}
Để đạt được mục đích, dự án sẽ phải thực hiện các mục tiêu sau đây:

\begin{table}[H]
    \centering
    \begin{tabular}{|p{5cm}|p{9.5cm}|}
    \hline
    \centering \textbf{Các mục tiêu} 
    & \textbf{Mô tả chi tiết} \\
    \hline
    \centering Tối ưu hóa trải nghiệm chơi game  
    & -- Hệ thống tương tác với các NPC và vật phẩm phong phú, cho phép người chơi thu thập các loại thông tin khác nhau (lời khai, bằng chứng, manh mối).  \\
    & -- Phát triển "Sổ Tay" thông minh, giúp người chơi tự động sắp xếp và quản lý thông tin một cách hiệu quả. \\
    \hline
    \centering Xây dựng Gameplay và cốt truyện hấp dẫn
    & -- Xây dựng game loop và các hoạt động thú vị mà người chơi sẽ thực hiện trong game. \\
    & -- Xây dựng một cốt truyện ý nghĩa, rẽ nhãnh và có nhiều kết thúc phụ thuộc vào quyết định của người chơi. \\
    \hline
    \centering Thiết kế đồ họa, âm thanh
    & -- Vẽ các hoạt ảnh 2D sẽ xuất hiện trong game. \\
    & -- Thiết kế âm thanh và âm nhạc giúp game thêm sống động và hấp dẫn. \\
    \hline
    \centering Đảm bảo hiệu suất và khả năng mở rộng
    & -- Tối ưu hóa game để chạy mượt mà trên nhiều nền tảng, nhiều thiết bị khác nhau. \\
    & -- Xây dựng cấu trúc dự án linh hoạt, dễ dàng thêm các vụ án, nhân vật và cơ chế mới trong tương lại mà không ảnh hưởng tới hiệu suất. \\
    \hline
    \centering Thương mại hóa
    & -- Đảm bảo game có thể tiếp cận thị trường quốc tế, thu hút người chơi từ toàn thế giới. \\
    \hline
    \end{tabular}
\end{table}

\subsection{Phạm vi dự án}
\subsubsection{Yếu tố cốt lõi}
\begin{enumerate}
    \item \textbf{Hệ thống gameplay cốt lõi}
    \begin{itemize}
        \item \textbf{Cơ chế điều tra và thu thập thông tin:}
        Cho phép người chơi tìm kiếm, thu thập và kiểm tra các bằng chứng tại hiện trường vụ án.

        \item \textbf{Cơ chế tương tác với NPC:}
        Cho phép người chơi thẩm vấn, ghi lại lời khai của các nhân vật.

        \item \textbf{Hệ thống quản lí thông tin (Sổ Tay):}
        Tự động lưu trữ, sắp xếp bằng chứng và các lời khai.

        \item \textbf{Cơ chế suy luận và phá án:}
        Cho phép người chơi xâu chuỗi các manh mối, mắt xích để đưa ra kết luận cuối cùng.
    \end{itemize}

    \item \textbf{Cốt truyện ban đầu}
    \begin{itemize}
        \item Xây dựng một vụ án hoàn chỉnh bao gồm các địa điểm, nhân vật và các sự kiện.
        
        \item Một số lượng NPC đủ để người chơi tương tác và thu thập thông tin.
    \end{itemize}

    \item \textbf{Hệ thống đồ họa, âm thanh}
    \begin{itemize}
        \item Hệ thống đồ họa 2D và UI/UX thân thiện với người dùng.

        \item Hệ thống âm thanh và nhạc nền cơ bản.
    \end{itemize}

    \item \textbf{Phát hành trên nền tảng PC (Window, MacOS, Linux).}
\end{enumerate}

\subsubsection{Yếu tố mở rộng}
\begin{enumerate}
    \item \textbf{Cải thiện hệ thống đồ họa và âm thanh}.
    \begin{itemize}
        \item Phát triển hệ thống đồ họa đẹp và bắt mắt. 
        
        \item Phát triển hệ thống âm thanh môi trường chân thật, âm nhạc cuốn hút. Cho phép điều chỉnh âm lượng tổng, âm thanh môi trường, âm nhạc và sound effects. 
    \end{itemize}

    \item \textbf{Mở rộng cốt truyện.}
    \begin{itemize}
        \item Xây dựng thêm các vụ án nhỏ bên cạnh vụ án chính.

        \item Thêm các địa điểm, NPC và vật phẩm.
    \end{itemize}

    \item \textbf{Thêm cơ chế cho gameplay.}
    \item \textbf{Phát hành bản mobile (iOS, android).}
\end{enumerate}

\subsection{Các bên liên quan}

\begin{itemize}
    \item \textbf{Nhà phát hành (Publisher)}: 
    \begin{itemize}
        \item Yêu cầu game có tiềm năng thương mại, phù hợp với thị trường quốc tế.
        \item Đảm bảo thời gian phát hành đúng tiến độ và có kế hoạch marketing rõ ràng.
    \end{itemize}

    \item \textbf{Người chơi mục tiêu (Target Players)}:
    \begin{itemize}
        \item Mong muốn trải nghiệm cốt truyện sâu sắc, có chiều sâu tâm lý và yếu tố siêu thực.
        \item Giao diện trực quan, điều khiển mượt mà và âm thanh tạo cảm giác ám ảnh.
    \end{itemize}

    \item \textbf{Nhà đầu tư (Investors)}:
    \begin{itemize}
        \item Yêu cầu báo cáo tiến độ định kỳ, minh bạch về chi phí và tiềm năng sinh lời.
        \item Ưu tiên các yếu tố sáng tạo có thể tạo điểm nhấn trên thị trường indie.
    \end{itemize}

    \item \textbf{Đội ngũ phát triển (Development Team)}:
    \begin{itemize}
        \item Cần môi trường làm việc linh hoạt, công cụ hỗ trợ hiệu quả và định hướng rõ ràng từ ban quản lý.
        \item Mong muốn được thể hiện cá tính nghệ thuật trong thiết kế hình ảnh và âm thanh.
    \end{itemize}

    \item \textbf{Cộng đồng game thủ (Gaming Community)}:
    \begin{itemize}
        \item Kỳ vọng được cập nhật thông tin thường xuyên, có cơ hội đóng góp ý kiến qua bản thử nghiệm.
        \item Mong muốn game có chiều sâu, có thể phân tích và thảo luận trên các diễn đàn.
    \end{itemize}

    
\end{itemize}
